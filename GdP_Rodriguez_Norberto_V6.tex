\documentclass[
11pt, % The default document font size, options: 10pt, 11pt, 12pt
%codirector, % Uncomment to add a codirector to the title page
]{charter} 




% El títulos de la memoria, se usa en la carátula y se puede usar el cualquier lugar del documento con el comando \ttitle
\titulo{Sistema de monitoreo para un bioterio} 

% Nombre del posgrado, se usa en la carátula y se puede usar el cualquier lugar del documento con el comando \degreename
%\posgrado{Carrera de Especialización en Sistemas Embebidos} 
\posgrado{Carrera de Especialización en Internet de las Cosas} 
%\posgrado{Carrera de Especialización en Intelegencia Artificial}
%\posgrado{Maestría en Sistemas Embebidos} 
%\posgrado{Maestría en Internet de las cosas}

% Tu nombre, se puede usar el cualquier lugar del documento con el comando \authorname
\autor{LSI Norberto Antonio Rodríguez} 

% El nombre del director y co-director, se puede usar el cualquier lugar del documento con el comando \supname y \cosupname y \pertesupname y \pertecosupname
\director{Mg. Ing. Gustavo Zocco}
\pertenenciaDirector{FIUBA} 
% FIXME:NO IMPLEMENTADO EL CODIRECTOR ni su pertenencia
%\codirector{John Doe} % para que aparezca en la portada se debe descomentar la opción codirector en el documentclass
%\pertenenciaCoDirector{FIUBA}
\codirector{Nombre del Codirector} % para que aparezca en la portada se debe descomentar la opción codirector en el documentclass
\pertenenciaCoDirector{pertenencia}

% Nombre del cliente, quien va a aprobar los resultados del proyecto, se puede usar con el comando \clientename y \empclientename
\cliente{Dr. Juan Rosa}
\empresaCliente{Instituto de Medicina Regional - UNNE}

% Nombre y pertenencia de los jurados, se pueden usar el cualquier lugar del documento con el comando \jurunoname, \jurdosname y \jurtresname y \perteunoname, \pertedosname y \pertetresname.
\juradoUno{Nombre y Apellido (1)}
\pertenenciaJurUno{pertenencia (1)} 
\juradoDos{Nombre y Apellido (2)}
\pertenenciaJurDos{pertenencia (2)}
\juradoTres{Nombre y Apellido (3)}
\pertenenciaJurTres{pertenencia (3)}
 
\fechaINICIO{21 de junio de 2022}		%Fecha de inicio de la cursada de GdP \fechaInicioName
\fechaFINALPlan{16 de agosto de 2022} 	%Fecha de final de cursada de GdP
\fechaFINALTrabajo{15 de mayo de 2023}	%Fecha de defensa pública del trabajo final


\begin{document}

\maketitle
\thispagestyle{empty}
\pagebreak


\thispagestyle{empty}
{\setlength{\parskip}{0pt}
\tableofcontents{}
}
\pagebreak


\section*{Registros de cambios}
\label{sec:registro}


\begin{table}[ht]
\label{tab:registro}
\centering
\begin{tabularx}{\linewidth}{@{}|c|X|c|@{}}
\hline
\rowcolor[HTML]{C0C0C0} 
Revisión & \multicolumn{1}{c|}{\cellcolor[HTML]{C0C0C0}Detalles de los cambios realizados} & Fecha      \\ \hline
0      & Creación del documento.                                 &\fechaInicioName \\ \hline
1      & Se completa hasta el punto 5 inclusive.                 & 3 de julio de 2022 \\ \hline
2      & Se realizaron correcciones en distintos capítulos y se completó hasta el punto 9 inclusive.      & 9 de julio de 2022 \\ \hline
3      & Se realizaron correcciones en distintos capítulos y se completó hasta el punto 12 inclusive.      & 24 de julio de 2022 \\ \hline
4      & Se realizaron correcciones en los capítulos 11 y 12, también se completó hasta el punto 15 inclusive.      & 1 de agosto de 2022 \\ \hline
5      & Se realizaron correcciones mínimas.      & 4 de agosto de 2022 \\ \hline
6      & Se realizaron correcciones en el uso de mayúsculas.      & 8 de agosto de 2022 \\ \hline

%2      & Se completa hasta el punto 7 inclusive
%		  Se puede agregar algo más \newline
%		  En distintas líneas \newline
%		  Así                                                    & dd/mm/aaaa \\ \hline
%3      & Se completa hasta el punto 11 inclusive                & dd/mm/aaaa \\ \hline
%4      & Se completa el plan	                                 & dd/mm/aaaa \\ \hline
\end{tabularx}
\end{table}

\pagebreak



\section*{Acta de constitución del proyecto}
\label{sec:acta}

\begin{flushright}
Buenos Aires, \fechaInicioName
\end{flushright}

\vspace{2cm}

%Por medio de la presente se acuerda con el Lic. \authorname\hspace{1px} que su Trabajo Final de la \degreename\hspace{1px} se titulará ``\ttitle'', consistirá esencialmente en \textcolor{red}{la implementación de un prototipo de un sistema de monitoreo de las condiciones ambientales de un bioterio}, y tendrá un presupuesto preliminar estimado de \textcolor{red}{600} hs de trabajo y \textcolor{red}{\$75.000}, con fecha de inicio \fechaInicioName\hspace{1px} y fecha de presentación pública \fechaFinalName.%

Por medio de la presente se acuerda con el Lic. \authorname\hspace{1px} que su Trabajo Final de la \degreename\hspace{1px} se titulará ``\ttitle'', consistirá esencialmente en {la implementación de un prototipo de un sistema de monitoreo de las condiciones ambientales de un bioterio}, y tendrá un presupuesto preliminar estimado de {600} hs de trabajo y {ARS 1.002.977}, con fecha de inicio \fechaInicioName\hspace{1px} y fecha de presentación pública \fechaFinalName.

Se adjunta a esta acta la planificación inicial.

\vfill

% Esta parte se construye sola con la información que hayan cargado en el preámbulo del documento y no debe modificarla
\begin{table}[ht]
\centering
\begin{tabular}{ccc}
\begin{tabular}[c]{@{}c@{}}Dr. Ing. Ariel Lutenberg \\ Director posgrado FIUBA\end{tabular} & \hspace{2cm} & \begin{tabular}[c]{@{}c@{}}\clientename \\ \empclientename \end{tabular} \vspace{2.5cm} \\ 
\multicolumn{3}{c}{\begin{tabular}[c]{@{}c@{}} \supname \\ Director del Trabajo Final\end{tabular}} \vspace{2.5cm} \\
%\begin{tabular}[c]{@{}c@{}}\jurunoname \\ Jurado del Trabajo Final\end{tabular}     &  & \begin{tabular}[c]{@{}c@{}}\jurdosname\\ Jurado del Trabajo Final\end{tabular}  \vspace{2.5cm}  \\
%\multicolumn{3}{c}{\begin{tabular}[c]{@{}c@{}} \jurtresname\\ Jurado del Trabajo Final\end{tabular}} \vspace{.5cm}                                                                     
\end{tabular}
\end{table}




\section{1. Descripción técnica-conceptual del proyecto a realizar}
\label{sec:descripcion}

Un bioterio es un área artificial controlada donde se alojan animales o plantas bajo ciertas condiciones que permitan el desarrollo de la especie para guarda, cría o fines experimentales. Se deben registrar y estudiar las variables ambientales del laboratorio en general y de los bioterios en particular. Estas variables son: temperatura, humedad, cantidad de luz y CO2.

Estas áreas artificiales pueden utilizarse en nuevas disciplinas, por ejemplo, la ecoepidemiología, la cual supone un enfoque actual integrador del estudio de las enfermedades transmitidas a los animales, vegetales y humanos. Involucra los aspectos ecológicos y epidemiológicos denominados ecoepidemiológicos.

En el bioterio se tendrán artrópodos. Son animales invertebrados que actúan como uno de los agentes transmisores de enfermedades. Lo hacen por acción directa como vector o indirecta por acción traumática e irritante de su picadura. Este proceso es independiente a su rol en el ciclo de vida del agente etiológico viral, bacteriano o parasitario.

Los ciclos reproductivos de los artrópodos juegan un papel fundamental en la epidemiología de las enfermedades transmisibles. Hay diferentes causas que pueden afectar la transmisión de una enfermedad: las tasas de fecundidad y de mortalidad, la densidad de la población, su distribución por edades, la variación genética y la tasa de migración.  Algunas de estas variables pueden ser controladas en condiciones experimentales como la temperatura, humedad, ciclos de luz y ventilación como factores abióticos esenciales. Los ambientes especialmente diseñados, denominados insectarios, bioterios o animalarios, son la base de diversos estudios (Insect Morphology and Phylogeny, 2014).

Los artrópodos tienen varios estados en su ciclo de vida que están relacionados con su metamorfosis. La misma puede ser completa (holometábolos): huevo, larva, pupa e imago. O bien incompleta (hemimetábolos) huevos, ninfas, imagos y con clasificaciones intermedias.  Cada uno de los estados tienen requerimientos de temperatura, humedad o inundación.  

Los artrópodos hematófagos presentan limitaciones al momento de mantenerlos o de desarrollar su biología. Dependen primero, de la fuente de alimentación sanguínea y otros nutrientes.  También influye el rango de temperatura, que en general oscila entre 24-26 ºC y 80-95 \% de humedad relativa, con fotoperíodos mediante ciclos de luz-oscuridad de 12 horas e intensidad de luz y tipo de luz variable según requerimientos (Arthropod Containment Guidelines, 2019).

Otro factor importante a tener en cuenta con los artrópodos es que algunos requieren agua como parte del ciclo de vida; como los culícidos (mosquitos), simúlidos y ceratopogónidos (carachai, jejenes y polvorines). Todos dependen de la calidad del agua y la temperatura ambiente. A mayor temperatura el ciclo se acorta y, por el contrario, a menor temperatura se prolonga; algunos casos incluso entran en latencia (diapausa).

Las larvas pueden ser acuáticas o terrestres. Se deben considerar la humedad relativa ambiental y requerimientos nutricionales diferentes. En el caso de las pupas o crisálidas, dependiendo de la especie de vectores, la temperatura es importante debido a que es un estado de transformación y no de alimentación como en la anterior. Los imagos eclosionarán en un ambiente controlado, pero no significa que continúen el ciclo de reproducción y generen colonias parentales o bien filiales de la misma especie (Insect Morphology and Phylogeny, 2014).

El estudio de la biología en condiciones experimentales solo puede aplicarse a determinados artrópodos debido a las limitaciones propias del ciclo natural según experiencias preliminares.  Los resultados de estas observaciones permiten investigar aspectos referentes a los mecanismos de transmisión de patógenos, las comparaciones genéticas de poblaciones, la sistemática y la susceptibilidad a los insecticidas, entre otros. Sin embargo, existe gran dificultad de mantener a la mayoría de las colonias durante más de unas pocas generaciones lo que impide contar con tablas de vida. Esta situación limita la comprensión del comportamiento de las colonias en la naturaleza (Insect Morphology and Phylogeny, 2014).

El entorno de una colonia no puede duplicar las condiciones de nutrición, temperatura y humedad. La consecuencia será la afectación en el comportamiento de los adultos, la mortalidad de los huevos, el período de crecimiento de las larvas y la mortalidad de los adultos. No obstante, los datos de laboratorio proporcionan información de referencia esencial sobre el potencial reproductivo de algunas especies sometidas a experiencia como mosquitos y flebótomos (Diptera: Psychodidae: Phlebotominae). 

Para estudios en biología de poblaciones de flebótomos en el campo, estas observaciones proporcionan pautas para aproximar los límites de los comportamientos biológicos esperados. Lo cual permite diseñar estrategias de control y de comprensión de comportamientos aplicables al desarrollo de mapas de riesgo epidemiológico (Escovar et al., 2004, Bueno Marí et al., 2015).


El presente proyecto se destaca especialmente por incorporar una aplicación y un dispositivo que contendrá sensores capaces de registrar la temperatura, humedad, luminosidad y saturación de CO2 del bioterio. Podrá ser usado en cada uno de los estados larvarios según las variables descriptas en los párrafos anteriores. Facilitará el registro, sistematización, alertas y presentación de datos periódicos en un tablero. Esto fortalecerá el proceso de investigación en todas sus etapas.

En la figura 1 se presenta el diagrama general del sistema. Se observa que los dispositivos estarán instalados en los bioterios con conexión Wi-Fi a Internet enviando la telemetría hacia el servidor del Instituto de Medicina Regional. Al mismo tiempo los investigadores pueden observar los datos recopilados.

\begin{figure}[htpb]
\centering 
\includegraphics[width=.75\textwidth]{./Figuras/figura1.png}
%\includegraphics[width=.5\textwidth]{./Figuras/figura1.png}
\caption{Diagrama general del sistema.}
\label{fig:diagBloques}
\end{figure}

\vspace{25px}

\section{2. Identificación y análisis de los interesados}
\label{sec:interesados}

\begin{table}[ht]
%\caption{Identificación de los interesados}
%\label{tab:interesados}
\begin{tabularx}{\linewidth}{@{}|l|X|X|l|@{}}
\hline
\rowcolor[HTML]{C0C0C0} 
Rol           & Nombre y Apellido & Organización 	& Puesto 	\\ \hline
Cliente       & \clientename      &\empclientename	& Investigador \\ \hline
Responsable   & \authorname       & FIUBA        	& Alumno 	\\ \hline
Orientador    & \supname	      & \pertesupname 	& Director Trabajo final \\ \hline
\end{tabularx}
\end{table}

\begin{itemize}
	\item Cliente: Juan Rosa es riguroso y exigente en los detalles.
	\item Responsable: Norberto Rodríguez, único personal en el equipo de desarrollo.
\end{itemize}


\section{3. Propósito del proyecto}
\label{sec:proposito}
El propósito de este proyecto es desarrollar un dispositivo para el monitoreo de las condiciones ambientales del bioterio, que incluya el registro de la telemetría, sistematización u organización de los datos recolectados, presentación de alertas vía SMTP a los investigadores y análisis de información por medio de un tablero de mandos.

\section{4. Alcance del proyecto}
\label{sec:alcance}

El proyecto incluye el diseño, desarrollo e implementación del sistema en conjunto con un prototipo de dispositivo instalado en el bioterio.

Se desarrollarán las siguientes actividades:
\begin{itemize}
	\item Confección de un prototipo de dispositivo encargado de la telemetría.
	\item Configuración del servidor Ubuntu server GNU/Linux distribución basada en Debian.
	\item Instalación de la aplicación web.
	\item Instalación y configuración de la base de datos.
	\item Confección de un dashboard.
	\item Confección del módulo de alertas vía SMTP.
\end{itemize}


El presente proyecto no incluye:
\begin{itemize}
	\item Provisión de la infraestructura de datos e Internet.
	\item Certificaciones ante autoridades competentes.
	\item Desarrollos para la automatización del bioterio.
\end{itemize}

\section{5. Supuestos del proyecto}
\label{sec:supuestos}

Para el desarrollo del presente proyecto se supone que:
\begin{itemize}
	\item Los bioterios tendrán acceso a Internet.
	\item Se contará con los recursos humanos necesarios por parte del Instituto de Medicina Regional para llevar a cabo el proyecto.
	\item El Instituto de Medicina Regional brindará acceso a los servidores para la instalación y configuración de la aplicación.
	\item Para la puesta en producción se utilizará el servidor de correo electrónico del instituto.
	\item Existe la disponibilidad de los componentes necesarios para el desarrollo del prototipo.
\end{itemize}

\section{6. Requerimientos}
\label{sec:requerimientos}

Los requerimientos identificados son:
\begin{enumerate}
	\item Requerimientos de hardware de los nodos
		\begin{enumerate}
			\item Cada nodo debe estar conformado por un microcontrolador.
			\item El nodo debe ser capaz de medir la temperatura y humedad.
			\item El nodo debe ser capaz de medir la Luz que reciben los artrópodos.
			\item El nodo debe ser capaz de medir CO2 en el bioterio.
			\item Fuente de alimentación debe estar incluida.
			\item El nodo debe poseer un módulo Wi-Fi para la comunicación.
		\end{enumerate}
	\item Requerimientos de software
		\begin{enumerate}
			\item El sistema debe permitir configurar los parámetros para las alertas.
			\item El usuario debe poder seleccionar el nodo a visualizar en el tablero de mandos.
			\item El sistema debe permitir asignar un nodo a un bioterio en particular.
		\end{enumerate}
	\item Requerimientos de presentación
		\begin{enumerate}
			\item El sistema debe proveer un tablero de mandos.
			\item El usuario debe poder visualizar la historia de la telemetría para un posterior análisis cuando se evalúe el desempeño de la colonia.
		\end{enumerate}
	\item Requerimientos de alertas
		\begin{enumerate}
			\item El sistema debe notificar al investigador sobre cambios en la telemetría mediante el envío de un correo electrónico.
			\item El usuario debe poder observar alertas en el tablero de mandos.
		\end{enumerate}
	\item Requerimientos de almacenamiento en la base de datos
		\begin{enumerate}
			\item Se debe almacenar la lectura de temperatura.
			\item Se debe almacenar la lectura de humedad.
			\item Se debe almacenar la lectura de cantidad de luz.
			\item Se debe almacenar la lectura de CO2.
			\item Se debe almacenar la lectura de fecha y hora de la telemetría.
		\end{enumerate}
\end{enumerate}

\section{7. Historias de usuarios (\textit{Product backlog})}
\label{sec:backlog}

Como criterio de ponderación se utilizó la serie de Fibonacci.

\begin{enumerate}
	\item Dificultad
		\begin{enumerate}
			\item Baja - peso 1.
			\item Media - peso 3.
			\item Alta - peso 5.
		\end{enumerate}
	\item Complejidad
		\begin{enumerate}
			\item Baja - peso 1.
			\item Media - peso 5.
			\item Alta - peso 13.
		\end{enumerate}
	\item Riesgo o incertidumbre
		\begin{enumerate}
			\item Bajo - peso 2.
			\item Medio - peso 3.
			\item Alto - peso 5.
		\end{enumerate}
\end{enumerate}

\begin{itemize}
	\item Como investigador deseo poder visualizar la temperatura del bioterio para que el cambio de temperatura no comprometa la colonia de artrópodos y tomar acciones preventivas, por ejemplo el corte de energía eléctrica que afectan a los aires acondicionados.		
	Ponderación: 13.
	
	Dificultad: media (3).
	
	Complejidad: media (5).
	
	Riesgo: medio (3).
	
	(3 + 5 + 3 = 11 valor siguiente de la serie Fibonacci 13).

	\item Como investigador deseo poder visualizar la humedad del bioterio para que no se vea comprometida la colonia de artrópodos.
	Ponderación: 13.
	
	Dificultad: media (3).
	
	Complejidad: media (5).
	
	Riesgo: medio (3).
	
	(3 + 5 + 3 = 11 valor siguiente de la serie Fibonacci 13).
	
	\item Como investigador deseo poder visualizar la cantidad de luz recibida para  que no afecte a la colonia de artrópodos a largo plazo.
	Ponderación: 21.
	
	Dificultad: alta (5).
	
	Complejidad: medio (5).
	
	Riesgo: medio (3).
	
	(5 + 5 + 3 = 13 valor siguiente de la serie Fibonacci 21).
	
	\item Como investigador deseo poder visualizar la cantidad de CO2 del bioterio para que no se vea comprometida la colonia de artrópodos.
	Ponderación: 13.
	
	Dificultad: baja (1).
	
	Complejidad: media (5).
	
	Riesgo: bajo (2).
	
	(1 + 5 + 2 = 8 valor siguiente de la serie Fibonacci 13).
	
	\item Como investigador deseo poder consultar los datos registrados con anterioridad para medir o comparar su impacto en la evolución de la colonia.
	Ponderación: 34.
	
	Dificultad: alta (5).
	
	Complejidad: alta (13).
	
	Riesgo: medio (3).
	
	(5 + 13 + 3 = 21 valor siguiente de la serie Fibonacci 34).
	
	\item Como investigador deseo recibir alertas por correo electrónico ante cambios en las condiciones del bioterio y así poder tomar medidas correctivas.
	Ponderación: 34.
	
	Dificultad: alta (5).
	
	Complejidad: alta (13).
	
	Riesgo: alto (5).
	
	(5 + 13 + 5 = 23 valor siguiente de la serie Fibonacci 34).
	
	\item Como investigador deseo poder visualizar un tablero de mandos independientemente de mi ubicación para supervisar el bioterio.
	Ponderación: 34.
	
	Dificultad: media (3).
	
	Complejidad: alta (13).
	
	Riesgo: alto (5).
	
	(3 + 13 + 5 = 21 valor siguiente de la serie Fibonacci 34).
	
\end{itemize}

\section{8. Se incluyen los siguietes entregables:}
\label{sec:entregables}

\begin{itemize}
	\item Manual de uso.
	\item Diagrama del sistema.
	\item Código fuente del firmware.
	\item Código fuente del la aplicación.
	\item Un prototipo de nodo.
	\item Informe de avance.
	\item Memoria escrita.
\end{itemize}

\section{9. Desglose del trabajo en tareas}
\label{sec:wbs}

\begin{enumerate}
\item Investigación y planificación del proyecto. (60 hs)
	\begin{enumerate}
	\item Seleccionar la arquitectura para el sistema. (5 hs)
	\item Investigar tecnologías IoT. (15 hs)
	\item Investigar sensores. (15 hs)
	\item Investigar microcontroladores. (15 hs)
	\item Investigar arquitecturas de microservicios. (10 hs)
	\end{enumerate}
\item Adquisición de componentes. (10 hs)
	\begin{enumerate}
	\item Consultar costos de los componentes. (5 hs)
	\item Realizar los pedidos a proveedores. (5 hs)
	\end{enumerate}
\item Instalación e implementación de la base de datos. (20 hs)
	\begin{enumerate}
	\item Instalación. (8 hs)
	\item Creación y configuración (12 hs)

	\end{enumerate}
\item Desarrollo del sistema - back-end. (120 hs)
	\begin{enumerate}
	\item Instalación de entorno de trabajo. (40 hs)
	\item Desarrollo de API. (40 hs)
	\item Desarrollo de alertas. (40 hs)
	\end{enumerate}
\item Desarrollo del sistema - front-end. (120 hs)
	\begin{enumerate}
	\item Desarrollo de acceso al sistema. (20 hs)
	\item Desarrollo del tablero de mandos. (45 hs)
	\item Desarrollo de visualización histórica de telemetría. (15 hs)
	\item Desarrollo de configuración de alertas. (25 hs)
	\item Desarrollo del menú de opciones. (5 hs)
	\item Desarrollo de ajustes de dispositivos. (5 hs)
	\item Desarrollo de ajustes de usuarios. (5 hs)
	\end{enumerate}
\item Desarrollo de firmware - módulo de temperatura y humedad. (85 hs)
	\begin{enumerate}
	\item Construcción del módulo e integración de componentes. (20 hs)
	\item Desarrollo del firmware para lectura de datos. (40 hs)
	\item Prueba de conexión con la aplicación. (25 hs)
	\end{enumerate}
\item Desarrollo de firmware - módulo de luz. (55 hs)
	\begin{enumerate}
	\item Construcción del módulo e integración de componentes. (10 hs)
	\item Desarrollo del firmware para lectura de datos. (30 hs)
	\item Prueba de conexión con la aplicación. (15 hs)
	\end{enumerate}
\item Desarrollo de firmware - módulo de CO2. (55 hs)
	\begin{enumerate}
	\item Construcción del módulo e integración de componentes. (10 hs)
	\item Desarrollo del firmware para lectura de datos. (30 hs)
	\item Prueba de conexión con la aplicación. (15 hs)
	\end{enumerate}
\item Testing. (30 hs)
	\begin{enumerate}
	\item Prueba de conectividad. (2 hs)
	\item Prueba de sensores. (3 hs)
	\item Prueba del tablero de mandos. (4 hs)
	\item Prueba de alertas. (4 hs)
	\item Verificación y validación del sistema completo. (10 hs)
	\item Prueba del prototipo en ambiente real. (7 hs)
	\end{enumerate}
\item Presentación del proyecto y cierre. (45 hs)
	\begin{enumerate}
	\item Escritura del informe de avance. (15 hs)
	\item Escritura de la memoria escrita. (15 hs)
	\item Escritura del plan de trabajo final. (15 hs)
	\end{enumerate}
\end{enumerate}

Cantidad total de horas: (600 hs)

\section{10. Diagrama de activity on node}
\label{sec:AoN}

%\begin{consigna}{red}
%Armar el AoN a partir del WBS definido en la etapa anterior. 

%La figura \ref{fig:AoN} fue elaborada con el paquete latex tikz y pueden consultar la siguiente referencia \textit{online}:

%\url{https://www.overleaf.com/learn/latex/LaTeX_Graphics_using_TikZ:_A_Tutorial_for_Beginners_(Part_3)\%E2\%80\%94Creating_Flowcharts}

%\end{consigna}

\begin{figure}[htpb]
\centering 
\includegraphics[height=1.49\textwidth]{./Figuras/AoN.png}
%\includegraphics[height=1.5\textwidth]{./Figuras/ActivityOnNode.png}
%\includegraphics[height=.85\textheight]{./Figuras/Gantt-2.png}
\caption{Diagrama en \textit{activity on node}.}
\label{fig:AoN}
\end{figure}

En la figura 2 se muestra el diagrama activity on node, la unidad de tiempo está expresada en horas. Las líneas en color rojo indica el camino crítico.



\section{11. Diagrama de Gantt}
\label{sec:gantt}

\begin{figure}[htpb]
\centering 
\includegraphics[width=.50\textheight]{./Figuras/Tareas.png}
\caption{Tabla de actividades.}
\label{fig:AoN}
\end{figure}

%\begin{figure}[htbp]
%\begin{center}
%\begin{ganttchart}{1}{12}
%  \gantttitle{2020}{12} \\
%  \gantttitlelist{1,...,12}{1} \\
%  \ganttgroup{Investigación y planificación del proyecto.}{1}{2} \\
%  \ganttbar{Seleccionar la arquitectura para el sistema.}{1}{2} \\
%  \ganttlinkedbar{Investigar tecnologías IoT.}{3}{7} \ganttnewline
%  \ganttmilestone{Milestone o hito}{7} \ganttnewline
%  \ganttbar{Final Task}{8}{12}
%  \ganttlink{elem2}{elem3}
%  \ganttlink{elem3}{elem4}  
%  \ganttgroup{Adquisición de componentes.}{1}{7} \\
%  \ganttbar{Task 1}{1}{2} \\
%  \ganttlinkedbar{Task 2}{3}{7} \ganttnewline
%  \ganttmilestone{Milestone o hito}{7} \ganttnewline
%  \ganttbar{Final Task}{8}{12}
%  \ganttlink{elem2}{elem3}
%  \ganttlink{elem3}{elem4}
%\end{ganttchart}
%\end{center}
%\caption{Diagrama de gantt de ejemplo}
%\label{fig:gantt}
%\end{figure}

%\begin{consigna}{red}
%Existen muchos programas y recursos \textit{online} para hacer diagramas de gantt, %entre los cuales destacamos:
En la figura 3 se puede ver la tabla de actividades. En las figuras 4 y 5 se observa el diagrama de Gantt. Se calcula una jornada diaria de trabajo de 8hs.

%\begin{figure}[htbp]
%\begin{center}
%\begin{ganttchart}{1}{12}
%  \gantttitle{2020}{12} \\
%  \gantttitlelist{1,...,12}{1} \\
%  \ganttgroup{Group 1}{1}{7} \\
%  \ganttbar{Task 1}{1}{2} \\
%  \ganttlinkedbar{Task 2}{3}{7} \ganttnewline
%  \ganttmilestone{Milestone o hito}{7} \ganttnewline
%  \ganttbar{Final Task}{8}{12}
%  \ganttlink{elem2}{elem3}
%  \ganttlink{elem3}{elem4}
%\end{ganttchart}
%\end{center}
%\caption{Diagrama de gantt de ejemplo}
%\label{fig:gantt}
%\end{figure}

\begin{landscape}
\begin{figure}[htpb]
\centering 
%\includegraphics[height=.85\textheight]{./Figuras/Gantt-2.png}
%\includegraphics[width=1\textheight]{./Figuras/Gantt1.png}
\includegraphics[height=.85\textheight]{./Figuras/DiagramaGantt_1.png}
\caption{Diagrama de Gantt parte 1/2.}
\label{fig:diagGantt}
\end{figure}
\end{landscape}
%\end{consigna}

\begin{landscape}
\begin{figure}[htpb]
\centering 
%\includegraphics[height=.85\textheight]{./Figuras/Gantt-2.png}
%\includegraphics[width=1\textheight]{./Figuras/Gantt1.png}
%\includegraphics[height=.57\textheight]{./Figuras/Gantt2.png}
\includegraphics[height=.90\textheight]{./Figuras/DiagramaGantt_2.png}
\caption{Diagrama de Gantt parte 2/2.}
\label{fig:diagGantt}
\end{figure}
\end{landscape}



\section{12. Presupuesto detallado del proyecto}
\label{sec:presupuesto}

A continuación se presentan los costos en pesos argentinos (ARS).
% Please add the following required packages to your document preamble:
% \usepackage[table,xcdraw]{xcolor}
% If you use beamer only pass "xcolor=table" option, i.e. \documentclass[xcolor=table]{beamer}
\begin{table}[htpb]
\centering
\begin{tabular}{|lccc|}
\hline
\rowcolor[HTML]{C0C0C0} 
\multicolumn{4}{|c|}{\cellcolor[HTML]{C0C0C0}{\color[HTML]{000000} COSTOS DIRECTOS}}                                                                                                                                                                                                      \\ \hline
\rowcolor[HTML]{C0C0C0} 
\multicolumn{1}{|c|}{\cellcolor[HTML]{C0C0C0}{\color[HTML]{000000} Descripción}} & \multicolumn{1}{c|}{\cellcolor[HTML]{C0C0C0}{\color[HTML]{000000} Cantidad}} & \multicolumn{1}{c|}{\cellcolor[HTML]{C0C0C0}{\color[HTML]{000000} Valor unitario}} & {\color[HTML]{000000} Valor Total} \\ \hline
\multicolumn{1}{|l|}{Horas de ingeniería}                                        & \multicolumn{1}{c|}{600}                                                     & \multicolumn{1}{c|}{1560,00}                                                        & 936.000,00                         \\ \hline
\multicolumn{1}{|l|}{Placa esp-wroom-32}                                         & \multicolumn{1}{c|}{3}                                                       & \multicolumn{1}{c|}{6.409,00}                                                      & 19.227,00                          \\ \hline
\multicolumn{1}{|l|}{Sensor de temperatura y humedad}                            & \multicolumn{1}{c|}{3}                                                       & \multicolumn{1}{c|}{1.150,00}                                                      & 3.450,00                            \\ \hline
\multicolumn{1}{|l|}{Sensor de luz}                                              & \multicolumn{1}{c|}{3}                                                       & \multicolumn{1}{c|}{1.050,00}                                                      & 3.150,00                           \\ \hline
\multicolumn{1}{|l|}{Sensor de CO2}                                              & \multicolumn{1}{c|}{3}                                                       & \multicolumn{1}{c|}{1.100,00}                                                      & 3.300,00                           \\ \hline
\multicolumn{1}{|l|}{Fuente de alimentación}                                     & \multicolumn{1}{c|}{3}                                                       & \multicolumn{1}{c|}{2.500,00}                                                      & 7.500,00                           \\ \hline
\multicolumn{1}{|l|}{Cable USB}                                                  & \multicolumn{1}{c|}{3}                                                       & \multicolumn{1}{c|}{950}                                                           & 2.850,00                           \\ \hline
\multicolumn{1}{|l|}{Case}                                                       & \multicolumn{1}{c|}{3}                                                       & \multicolumn{1}{c|}{2.500,00}                                                      & 7.500,00                           \\ \hline
\multicolumn{3}{|c|}{SUBTOTAL}                                                                                                                                                                                                                       & 982.977,00                         \\ \hline
\rowcolor[HTML]{C0C0C0} 
\multicolumn{4}{|c|}{\cellcolor[HTML]{C0C0C0}COSTOS INDIRECTOS}                                                                                                                                                                                                                           \\ \hline
\rowcolor[HTML]{C0C0C0} 
\multicolumn{1}{|c|}{\cellcolor[HTML]{C0C0C0}Descripción}                        & \multicolumn{1}{c|}{\cellcolor[HTML]{C0C0C0}Cantidad}                        & \multicolumn{1}{c|}{\cellcolor[HTML]{C0C0C0}Valor unitario}                        & Valor total                        \\ \hline
\multicolumn{1}{|l|}{Transporte y movilidad}                                     & \multicolumn{1}{c|}{2}                                                       & \multicolumn{1}{c|}{10.000,00}                                                     & 20.000,00                          \\ \hline
\multicolumn{3}{|c|}{SUBTOTAL}                                                                                                                                                                                                                       & 20.000,00                          \\ \hline
\rowcolor[HTML]{C0C0C0} 
\multicolumn{3}{|c|}{\cellcolor[HTML]{C0C0C0}TOTAL}                                                                                                                                                                                                  & 1.002.977,00                         \\ \hline
\end{tabular}
\end{table}

%\begin{consigna}{red}
%Si el proyecto es complejo entonces separarlo en partes:
%\begin{itemize}
%	\item Un total global, indicando el subtotal acumulado por cada una de las áreas.%	\item El desglose detallado del subtotal de cada una de las áreas.
%\end{itemize}
%IMPORTANTE: No olvidarse de considerar los COSTOS INDIRECTOS.
%\end{consigna}

%\begin{table}[htpb]
%\centering
%\begin{tabularx}{\linewidth}{@{}|X|c|r|r|@{}}
%\hline
%\rowcolor[HTML]{C0C0C0} 
%\multicolumn{4}{|c|}{\cellcolor[HTML]{C0C0C0}COSTOS DIRECTOS} \\ \hline
%\rowcolor[HTML]{C0C0C0} Descripción &
%\multicolumn{1}{c|}{\cellcolor[HTML]{C0C0C0}Cantidad} &
%\multicolumn{1}{c|}{\cellcolor[HTML]{C0C0C0}Valor unitario} &
%\multicolumn{1}{c|}{\cellcolor[HTML]{C0C0C0}Valor total} \\ \hline 
%\multicolumn{1}{|c|}{Horas de ingeniería} & 
%\multicolumn{1}{c|}{600} &
%\multicolumn{1}{c|}{800,00} &
%\multicolumn{1}{c|}{480.000,00} \\ \hline 
%\multicolumn{1}{|c|}{Placa esp-wroom-32}  & 
%\multicolumn{1}{c|}{3} &
%\multicolumn{1}{c|}{6.409,00} &
%\multicolumn{1}{c|}{19.227,00} \\ \hline
%\multicolumn{1}{|c|}{Sensor de temperatura y humedad}  & 
%\multicolumn{1}{c|}{3} &
%\multicolumn{1}{c|}{1.150,00} &
%\multicolumn{1}{c|}{3.450,00} \\ \hline
%\multicolumn{1}{|c|}{Sensor de luz}  & 
%\multicolumn{1}{c|}{3} &
%\multicolumn{1}{c|}{1.050,00} &
%\multicolumn{1}{c|}{3.150,00} \\ \hline
%\multicolumn{1}{|c|}{Sensor de CO2}  & 
%\multicolumn{1}{c|}{3} &
%\multicolumn{1}{c|}{1.100,00} &
%\multicolumn{1}{c|}{3.300,00} \\ \hline
%\multicolumn{1}{|c|}{Fuente de alimentación}  & 
%\multicolumn{1}{c|}{3} &
%\multicolumn{1}{c|}{2.500,00} &
%\multicolumn{1}{c|}{7.500,00} \\ \hline
%\multicolumn{1}{|c|}{Cable USB}  & 
%\multicolumn{1}{c|}{3} &
%\multicolumn{1}{c|}{950,00} &
%\multicolumn{1}{c|}{2.850,00} \\ \hline
%\multicolumn{1}{|c|}{Case}  & 
%\multicolumn{1}{c|}{3} &
%\multicolumn{1}{c|}{2.500,00} &
%\multicolumn{1}{c|}{7.500,00} \\ \hline
%\multicolumn{3}{|c|}{SUBTOTAL} &
%\multicolumn{1}{c|}{526.977,00} \\ \hline
%\rowcolor[HTML]{C0C0C0} 
%\multicolumn{4}{|c|}{\cellcolor[HTML]{C0C0C0}COSTOS INDIRECTOS} \\ \hline
%\rowcolor[HTML]{C0C0C0} Descripción &
%\multicolumn{1}{c|}{\cellcolor[HTML]{C0C0C0}Cantidad} &
%\multicolumn{1}{c|}{\cellcolor[HTML]{C0C0C0}Valor unitario} &
%\multicolumn{1}{c|}{\cellcolor[HTML]{C0C0C0}Valor total} \\ \hline
%\multicolumn{1}{|c|}{Transporte y movilidad}  & 
%\multicolumn{1}{c|}{2} &
%\multicolumn{1}{c|}{10.000,00} &
%\multicolumn{1}{c|}{20.000,00} \\ \hline
%\multicolumn{3}{|c|}{SUBTOTAL} &
%\multicolumn{1}{c|}{20.000,00} \\ \hline
%\rowcolor[HTML]{C0C0C0}
%\multicolumn{3}{|c|}{TOTAL} &
%\multicolumn{1}{c|}{546.977,00} \\ \hline
%\end{tabularx}%
%\end{table}

\section{13. Gestión de riesgos}
\label{sec:riesgos}

En la sección actual se describirán los riesgos del proyecto cuantificados en un rango de 1 al 10 y su plan de contingencia correspondiente.

\begin{itemize}
	\item Severidad (S): cuánto más severo, más alto es el número.\\

	\item Ocurrencia (O): cuánto más posible sea  que ocurra, más alto es el número.\\
\end{itemize}   

%\begin{consigna}{red}
a) Identificación de los riesgos y estimación de sus consecuencias:
 
Riesgo 1: la falta de precesión de los sensores al momento de realizar las lecturas.
\begin{itemize}
	\item Severidad (9): si los sensores hacen lecturas muy distintas a las que se hacen manualmente en la actualidad.\\
	\item Probabilidad de ocurrencia (5): suelen tener alta calidad y fiabilidad.\\
\end{itemize}   

Riesgo 2: pérdida o daño de los nodos.
\begin{itemize}
	\item Severidad (9): generará una demora  que exigirá tener un reemplazo.
	\item Ocurrencia (2): por tratarse de un prototipo se suele tener cuidados adicionales.
\end{itemize}

Riesgo 3: dificultad con el uso de la tecnología seleccionada debido a la falta de experiencia.
\begin{itemize}
	\item Severidad (8): el desconocimiento de la tecnología puede llevar a una implementación deficiente.
	\item Ocurrencia (5): en proyectos de estas características se busca incorporar los avances tecnológicos disponibles.
\end{itemize}

Riesgo 4: no contar con un bioterio activo al momento de las pruebas.
\begin{itemize}
	\item Severidad (7): sin un bioterio activo es difícil estudiar el comportamiento del nodo en actividad y en condiciones normales de operación.
	\item Ocurrencia (2): en algunas ocasiones no está disponible.
\end{itemize}

Riesgo 5: que se cancele el proyecto con el Instituto de Medicina Regional - UNNE.
\begin{itemize}
	\item Severidad (9): al cancelar el proyecto se pierde el contacto con los investigadores.
	\item Ocurrencia (3): el proyecto es de interés para los investigadores actuales pero pueden haber factores externos al equipo de investigación.
\end{itemize}

Riesgo 6: incumplimiento de los plazos fijados.
\begin{itemize}
	\item Severidad (6): en caso de que algunas de las tareas requieran más horas de las destinadas y no hay disponibilidad para reasignar esas horas.
	\item Ocurrencia (5): si se desconoce el grado de dificultad de las tareas o un evento inesperado.
\end{itemize}

Riesgo 7: error de transmisión por la perdida de conexión a Internet.
\begin{itemize}
	\item Severidad (8): la no transmisión de los datos hacia el servidor va a generar que no se registren las lecturas de los sensores.
	\item Ocurrencia (3): en los equipos conectados 24/7 puede ocurrir que en algunos momentos no funcione correctamente.
\end{itemize}

Riesgo 8: las fallas del firmware pueden ocasionar que todo el nodo completo funcione de manera inesperada y se generen datos incorrectos.
\begin{itemize}
	\item Severidad (8): cuando se dan errores de firmware el producto deja de ser confiable.
	\item Ocurrencia (6): si la complejidad del código es alta y no se llega a probar en todas las situaciones posibles.
\end{itemize}

Riesgo 9: incumplimiento de lo pactado con el cliente.
\begin{itemize}
	\item Severidad (9): el sistema no contempla los requerimientos del cliente.
	\item Ocurrencia (3): existe una devolución del cliente en etapas anteriores a la entrega del producto final.
\end{itemize}

%Riesgo 10:
%\begin{itemize}
%	\item Severidad (S): 
%	\item Ocurrencia (O):
%\end{itemize}

b) Tabla de gestión de riesgos:      (El RPN se calcula como RPN = S x O)

%\begin{table}[htpb]
%\centering
%\begin{tabularx}{\linewidth}{@{}|X|c|c|c|c|c|c|@{}}
%\hline
%\rowcolor[HTML]{C0C0C0} 
%Riesgo & S & O & RPN & S* & O* & RPN* \\ \hline
%       &   &   &     &    &    &      \\ \hline
%       &   &   &     &    &    &      \\ \hline
%       &   &   &     &    &    &      \\ \hline
%       &   &   &     &    &    &      \\ \hline
%       &   &   &     &    &    &      \\ \hline
%\end{tabularx}%
%\end{table}

% Please add the following required packages to your document preamble:
% \usepackage[table,xcdraw]{xcolor}
% If you use beamer only pass "xcolor=table" option, i.e. \documentclass[xcolor=table]{beamer}
\begin{table}[htpb]
\centering
\begin{tabular}{|l|c|c|c|c|c|c|}
\hline
\rowcolor[HTML]{C0C0C0} 
\multicolumn{1}{|c|}{\cellcolor[HTML]{C0C0C0}Riesgo} & S  & O  & RPN                        & S* & O* & RPN*                       \\ \hline
1 Falta de precisión en las lecturas los sensores & 9  & 5  & 
\cellcolor[HTML]{CB0000}45 & 9  & 2  & 
\cellcolor[HTML]{009901}18  \\ \hline
2 Pérdida o daños de los nodos & 9 & 2 & 
\cellcolor[HTML]{009901}18 & - & - & - \\ \hline
3 Dificultad con el uso de la tecnología seleccionada  &8    &5    & 
\cellcolor[HTML]{CB0000}40 & 8 & 3 & 
\cellcolor[HTML]{009901}24 \\ \hline
4 No contar con un bioterio activo &7    &2    &
\cellcolor[HTML]{009901}14 & - & - & - \\ \hline
5 Que se cancele el proyecto &9 &3 &
\cellcolor[HTML]{009901}27 &- &- &- \\ \hline
6 Incumplimiento de los plazos fijados &6 &5 &
\cellcolor[HTML]{009901}30 &- &- &- \\ \hline
7 Error de transmisión por pérdida de conexión a Internet &8 &3 &
\cellcolor[HTML]{009901}24 &- &- &-\\ \hline
8 Las fallas del firmware &8 &6 & 
\cellcolor[HTML]{CB0000}48 &8 &4 &
\cellcolor[HTML]{009901}32 \\ \hline
9 Incumplimiento de lo pactado con el cliente &9 &3 &
\cellcolor[HTML]{009901}27 &- &- &- \\ \hline
\end{tabular}
\end{table}

Criterio adoptado: 
se tomarán medidas de mitigación en los riesgos cuyos números de RPN sean mayores a 39.

Nota: los valores marcados con (*) en la tabla corresponden luego de haber aplicado la mitigación.

c) Plan de mitigación de los riesgos que originalmente excedían el RPN máximo establecido:

Se trabajará un plan de mitigación para los riesgos 1, 3 y 8 por exceder el valor máximo establecido: 39.
 
Riesgo 1: se trabajarán con sensores estándares utilizados en las clases de CEIoT. De ser necesario se cambiarán los sensores por otros más precisos que existan en el mercado.
\begin{itemize}
	\item Severidad (9): se mantiene.
	\item Ocurrencia (2): se ajusta la precisión de las lecturas de los sensores.
\end{itemize}   


Riesgo 3: la tecnología seleccionada es la utilizada en distintos cursos de la CEIoT. Se realizará una mayor capacitación en las herramientas incluidas en el proyecto.
\begin{itemize}
	\item Severidad (8): se mantiene.
	\item Ocurrencia (3): se espera resolver de forma optima las dificultades encontradas.
\end{itemize}   
 
Riesgo 8: se realizarán tareas de prueba y depuración de los módulos involucrados. Hay toda una etapa de testing dedicada a probar el correcto funcionamiento del sistema.
\begin{itemize}
	\item Severidad (8): se mantiene.
	\item Ocurrencia (4): se busca un software libre de errores.
\end{itemize}   

%\end{consigna}


\section{14. Gestión de la calidad}
\label{sec:calidad}

%\begin{consigna}{red}
%Para cada uno de los requerimientos del proyecto indique:

\begin{itemize} 
\item 1 Requerimientos de hardware.
\begin{itemize}
	\item Verificación: se comprobará que los nodos estén conformados por un microcontrolador adecuado, los sensores correspondientes y demás componentes para su funcionamiento.
	\item Validación: se harán pruebas para confirmar que los nodos estén completos, envíen información adecuada y cumplan con las especificaciones el cliente.
\end{itemize}

\item 2 Requerimientos de software.
\begin{itemize}
	\item Verificación: se comprobará que los datos recibidos por el sistema se muestren en el tablero de mandos. De ser necesario el sistema debe responder con alertas y permitir la configuración de dispositivos o nodos.
	\item Validación: se harán pruebas en un bioterio bajo condiciones normales verificando el correcto funcionamiento del nodo, del sistema y la comunicación entre ellos.
\end{itemize}

\item 3 Requermientos de presentación.
\begin{itemize}
	\item Verificación: se comprobará que el tablero de mandos esté suministrando los datos correctos de acuerdo a la información recibida.
	\item Validación: se harán pruebas para demostrar que el tablero esté desplegando la información correcta.
\end{itemize}

\item 4 Requerimientos de alertas.
\begin{itemize}
	\item Verificación: se comprobará que el sistema envíe las alertas correspondientes de acuerdo a la configuración establecida.
	\item Validación: se harán pruebas para que el usuario reciba las notificaciones cuando los datos obtenidos desde el nodo indiquen que deben emitirse alertas.
\end{itemize}

\item 5 Requerimientos de almacenamiento en la base de datos.
\begin{itemize}
	\item Verificación: se comprobarán que los datos recibidos por el sistema desde los nodos se almacene correctamente en la base de datos.
	\item Validación: se harán pruebas para que los valores registrados por los sensores se guarden y se visualicen correctamente.
\end{itemize}

\end{itemize}

%Tener en cuenta que en este contexto se pueden mencionar simulaciones, cálculos, revisión de %hojas de datos, consulta con expertos, mediciones, etc.  Las acciones de verificación suelen %considerar al entregable como ``caja blanca'', es decir se conoce en profundidad su %funcionamiento interno.  En cambio, las acciones de validación suelen considerar al %entregable como ``caja negra'', es decir, que no se conocen los detalles de su funcionamiento %interno.

%\end{consigna}

\section{15. Procesos de cierre}    
\label{sec:cierre}

%\begin{consigna}{red}
Las actividades de los procesos de cierre serán llevadas a cabo por el responsable del plan de trabajo, LSI Norberto Antonio Rodríguez.

\begin{itemize}
	\item El grado de cumplimiento del cronograma original establecido y las tareas que pudieran generar mayor demanda de recursos, cumplimiento de los requerimientos, horas asignadas y presupuesto fijado.
	 
	\item Se observarán aquellas herramientas o técnicas que mejor resultados dieron al momento de llevar a cabo el cumplimiento del proyecto.
	
	\item Se realizará un acto de cierre para destacar y agradecer a todas aquellas personas que han aportado a las distintas tareas realizadas en el proyecto, al público, miembros que integran el jurado, al director del trabajo y docentes de la CEIoT.
\end{itemize}

%\end{consigna}


\end{document}
